\section{Diskussion}

\subsection{Messreihe}
Da der Messfehler der Waage $\pm 0.01g$ beträgt, aber die kleinst eingewogene Masse 4g war (Tabelle 2), kann dieser vernachlässigt werden. Es kann davon
ausgegangen werden das die Einwaagen nicht Fehlerbehaftet sind.

\subsection{Dissoziation von Essigsäure}
Bei Betrachtung von Tabelle 4 kann man sehr gut den linearen Zusammenhang zwischen Konzentration und der prozentuellen Dissoziation sehen. 
Wenn man die Konzentration um eine 10er Potenz erniedrigt, wird im gegenzug die Dissoziation fast verdreifacht. 
Weiter lässt sich durch die Ergebnisse von Lösung B3 und A5 erkennen, dass sauber gearbeitet wurde, da beide Lösungen theoretisch, wie auch praktisch die gleiche
Dissoziation aufweisen.

\subsection{Debye-Radius}
Hier war die Auswertung der Daten sehr zeitaufwändig, vorallem da die Formel des Debye-Radius nicht korrekt im Skript stand. 
Außerdem ist die Tatsache, der Korrelationskoeffizient bei beiden Temperaturen (Abbildung 2 und 4) gleich 1 ist und es somit keine Streuung der Werte um die Regressionsgerade 
gibt, unerwartet. Wenn man aber die errechneten Radien mit denen aus dem Skript vergleicht (Skript: Potentiometrie, Seite 8) sieht man, dass die Werte sich stark gleichen,
es kann also angenommen werden, dass kein Fehler vorliegt und die Werte extrem präzise sind, vielleicht aber nicht genau.


\subsection{Temperatureinfluss auf die Dissoziation und den Debye-Radius}

\subsubsection{Debye-Radius}

Bei der Berechnung des Quotienten X (Tabelle 10) stimmen die Ergebnisse noch relativ gut mit denen aus dem Skript (Seite 8) überein. Abbildung 5 lässt auch erkennen,
dass der Debye-Radius auch mit höherer Temperatur nicht schneller ansteigt, wenn die Konzentration erniedrigt wird. \\
Bei der darauffolgenden Berechnung des prozentuellen Unterschiedes zeigt sich aber im gegensatz zu den Daten im Skript, dass keine Änderung stattfindet und wenn
dann nur in einem so kleinem Maß, dass selbst Excel schwierigkeiten hat dies darzustellen (siehe Abbildung 6, y-Achse). \\
Um zu überprüfen ob nicht ein Fehler in der Berechnung liegt, wurde mit dem gleichen Verfahren untersucht, wie sich verschiedene Temperaturen auf den Verlauf der 
Dissoziation auswirken.

\subsubsection{Dissoziation}

Hier liefert die Methodik den erwarteten linearen Zusammenhang (Abbildung 9 und 10). 





