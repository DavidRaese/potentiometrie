\section{Messungen}

\subsection{Verwendete Geräte und Chemikalien}
\vspace{.5cm}
\begin{minipage}[H][3cm][t]{0.4\textwidth}
  \textbf{Chemikalien}
  \begin{itemize}
    \item Rotipuran (Essigsäure der Firma Roth, mit genau bestimmter Molalität)
    \item Destilliertes Wasser
  \end{itemize}
\end{minipage}
\begin{minipage}[H][3cm][t]{0.4\textwidth}
  \textbf{Geräte}
  \begin{itemize}
    \item Waage: Kern KB
    \item Wasserbad: Julabo 26
    \item pH-Meter: Mettler Toledo Seven Multi
  \end{itemize}
\end{minipage}

\subsection{Erstellung der Stammlösungen}
Um die Messreihen zu erstellen, die eine Konzentrationsbereich von 0.005 bis 0.00001 [mol/kg]
abdeckt, wurde aus einer 0.5 molalen Essigsäurelösung 3 Stammlösungen hergestellt.

\begin{table}[htbp]
    \centering
    \caption{Stammlösungen, aus 0.5molaler Essigsäure}
      \begin{tabular}{lr}
      \toprule
      \textbf{Stammlösung} & \multicolumn{1}{l}{\textbf{Konz [mmol/kg]}} \\
      \midrule
      I     & 50,012 \\
      II    & 5,015 \\
      III   & 0,5012 \\
      \bottomrule
      \end{tabular}%
    \label{tab:addlabel}%
\end{table}%
  
Aus diesen Stammlösungen wurde im nächsten Schritt die Messreihe erstellt.


\begin{table}[H]
    \centering
    \small
    \caption{Messreihe, nach Konzentration sortiert}
    \tabcolsep=0.11cm
      \begin{tabular}{lccccccc}
      \toprule
      \textbf{Lösung} & \multicolumn{1}{l}{\textbf{Stamm}} & \multicolumn{1}{l}{\textbf{Stamm [g]ideal}} & \multicolumn{1}{l}{\textbf{$H_2O$ [g]ideal}} & \multicolumn{1}{l}{\textbf{Stamm [g]real}} & \multicolumn{1}{l}{\textbf{$H_2O$ [g]real}} & \multicolumn{1}{l}{\textbf{Konz [mmol/kg]soll}} & \multicolumn{1}{l}{\textbf{Konz [mmol/kg]ist}} \\
      \midrule
      A1    & 50,012 & 50    & 0     & 50,06 & 0     & 50    & 50,01 \\
      A2    & 50,012 & 25    & 25    & 25,04 & 25,04 & 25    & 25,01 \\
      A3    & 50,012 & 15    & 35    & 15,1  & 35,05 & 15    & 15,06 \\
      B1    & 50,012 & 10    & 40    & 10,08 & 40,02 & 10    & 10,06 \\
      B2    & 50,012 & 6     & 44    & 6,08  & 44,2  & 6     & 6,05 \\
      A4    & 50,012 & 5     & 45    & 5,01  & 45,01 & 5     & 5,01 \\
      B3    & 5,015 & 10    & 40    & 10,02 & 40,03 & 1     & 1,00 \\
      A5    & 5,015 & 10    & 40    & 9,98  & 40,22 & 1     & 1,00 \\
      C1    & 5,015 & 8     & 42    & 8,03  & 42,05 & 0,8   & 0,80 \\
      C2    & 5,015 & 6     & 44    & 6,07  & 44,1  & 0,6   & 0,61 \\
      B4    & 5,015 & 5     & 45    & 5,04  & 45,01 & 0,5   & 0,51 \\
      C3    & 5,015 & 4     & 46    & 4,07  & 46,07 & 0,4   & 0,41 \\
      C4    & 0,5012 & 30    & 20    & 30,04 & 20,05 & 0,3   & 0,30 \\
      B5    & 0,5012 & 20    & 30    & 20    & 30    & 0,2   & 0,20 \\
      C5    & 0,5012 & 10    & 40    & 10,12 & 40,15 & 0,1   & 0,10 \\
      \bottomrule
      \end{tabular}%
    \label{tab:addlabel}%
\end{table}%F


\subsection{Erstellung der Messreihen}

Die Messserie wurde in drei Teile aufgeteilt, entsprechend den drei Stammlösungen.
Desweiteren wurde darauf geachtet, dass sich keine Konzentration mehr als einmal 
wiederholt. \\ Über diese beiden gleichen Konzentrationen kann man auch überprüfen, ob
die rechnerisch bestimmten Stammlösungen auch experimentell richtig sind. \\
Da die Konzentrationen eingewogen wurden und die Einwaagen von den idealen Werten abweichen, muss die reale Konzentration
noch aus den realen Einwaagen bestimmt werden.

\begin{align}
  b = \frac{Einwage \, Stamm [g] \cdot Konz. \, Stamm [g/mmol]}{Einwage \, Stamm + Einwage \, H_2O}
\end{align}

Für die Werte siehe Tabelle 2.

\begin{table}[H]
    \centering
    \caption{pH-Messung}
      \begin{tabular}{lccc}
      \toprule
      \textbf{Lösung} & \multicolumn{1}{l}{\textbf{Konz [mmol/kg] soll}} & \multicolumn{1}{l}{\textbf{pH bei 25$^\circ$C}} & \multicolumn{1}{l}{\textbf{pH bei 40$^\circ$C}} \\
      \midrule
      A1    & 50    & 3,03  & 3,06 \\
      A2    & 25    & 3,14  & 3,2 \\
      A3    & 15    & 3,25  & 3,32 \\
      B1    & 10    & 3,38  & 3,41 \\
      B2    & 6     & 3,5   & 3,53 \\
      A4    & 5     & 3,53  & 3,57 \\
      B3    & 1     & 3,91  & 3,94 \\
      A5    & 1     & 3,91  & 3,94 \\
      C1    & 0,8   & 3,96  & 4 \\
      C2    & 0,6   & 4,03  & 4,07 \\
      B4    & 0,5   & 4,07  & 4,11 \\
      C3    & 0,4   & 4,13  & 4,16 \\
      C4    & 0,3   & 4,21  & 4,26 \\
      B5    & 0,2   & 4,29  & 4,34 \\
      C5    & 0,1   & 4,5   & 4,54 \\
      \bottomrule
      \end{tabular}%
    \label{tab:addlabel}%
\end{table}%
  
  

Nach der Herstellung wurden die verschieden Konzentration mittels pH-Meter jeweils einmal bei
25$^\circ$C und bei 40$^\circ$C vermessen, dabei wird immer von der kleineren zu nächst höheren Konzentration gemessen.
Durch diese Methodik kann schnell gemessen werden ohne die Konzentration zu ändern.

