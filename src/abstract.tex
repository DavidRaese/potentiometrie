\section{Abstract}
In dieser Laborübung wurde der Zusammenhang zwischen Konzentration
und Dissoziation, sowie der Zusammenhang zwischen Konzentration und
Debye-Radius bei zwei verschiedenen Temperaturen (25$^\circ$C
und 40$^\circ$C) von Essigsäure untersucht. Bei der anschließenden Auswertung
lässt sich gut erkennen, wie die Dissoziation bei geringerer Konzentration steigt. Bei genügend hoher Verdünnung kann auch
aus einer schwachen Säure (hier Essigsäure) eine Starke gemacht werden. \\
Der weitere Teile der Auswertung konzentriert sich auf den Zusammenhang der Konzentration und des Debye-Radius, dieser 
vergrößert sich exponentiell, wenn sich die Konzentration verringert.   


